\section{Conclusions}
\label{Sec:Conclusion}

%We explore how the batch scheduler can efficiently allocate burst buffer to absorb three general types of IO operations:
%data staging in, application checkpointing, and data staging out.
%We propose a three-phase job model which is tailored to burst buffer's typical use cases.
%Burst-buffer-aware Cerberus is developed on the basis of the three-phase job model.
%We divide the scheduling problem into three sub-phases, and conquer them independently using dynamic programming based optimization.
%Simulation results show that Cerberus significantly improves both the application-level and the system-level performance.

In this study, we have presented Cerberus, a three-phase burst-buffer-aware scheduling design.
Our key contributions include a three-phase job model to describe user jobs
and new optimization-based job scheduling at each phase.
An event driven scheduling simulator named BBSim is developed for examining
Cerberus on burst buffer enabled systems, which is freely available for the community.
Our extensive trace based simulations demonstrate that
Cerberus can significantly accelerate job execution,
reduces job response time, and improves the average system throughput. 
