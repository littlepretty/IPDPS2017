\section{Background}
\label{Sec:Background}

% % %1. The appear of burst buffer
% % Burst buffer enabled I/O architecture can help catch up with
% % the ever-increasing computational performance and parallelism of HPC system.
% % Burst buffer nodes debut as a rescue by utilizing various types of memory,
% % for example, non-volatile random-access memory (NVRAM) and solid state drive (SSD).
% % The driver behind is these storage technologies' decreasing cost of bandwidth.
% % In practice, it can suits up with DataWarp application I/O accelerator\cite{DataWarp}.
% % Figure \ref{Fig:BBArchitecture} illustrates one possible architecture of
% % burst buffer enabled HPC system, adopted by Trinity System\cite{TrinitySystem}.
% % %The volume of data read/write may affect the architecture model of burst buffer.
% % In Trinity, burst buffer nodes is composed of I/O nodes and 2 PCIe SSD cards,
% % connected via totally 16 PCIe 3.0 interfaces.
% % Alternatively, many researchers proposed to distribute burst buffer 
% % on multiple layers of the memory hierarchy\cite{Romanus:CORR:15}.
% % For example, they may be deployed at local computer nodes, board in cabinet or I/O nodes.
% % We may also use burst buffer as intermediate storage system.

%1. The appear of burst buffer
Burst buffer enabled I/O architecture can help alleviate
the ever-increasing I/O pressure on HPC systems.
It debuts as a rescue by utilizing various new storage technologies,
such as non-volatile random-access memory (NVRAM) and solid state drive (SSD).
In practice, burst buffer suits up with DataWarp I/O accelerator\cite{DataWarp}.
Figure \ref{Fig:BBArchitecture} illustrates one burst-buffer-enabled architecture,
adopted by the Trinity System\cite{TrinitySystem}.
%The volume of data read/write may affect the architecture model of burst buffer.
In Trinity, the burst buffer node is composed of one I/O node and 2 PCIe SSD cards,
connected via 16 PCIe 3.0 interfaces.
Alternatively, many researchers have proposed to distribute burst buffer 
on multiple layers of the memory hierarchy\cite{Romanus:CORR:15}.
For example, they may be deployed on the local compute nodes, the board in the cabinet or the I/O nodes.
Burst buffer may also be utilized as the intermediate storage systems.

% % %2. Use cases of burst buffer
% % Regardless of the specific implementation, burst buffer nodes essentially augment
% % the I/O stack with a intermediate, productive offloading layer.
% % For example, an application's latest checkpoint can be pre-staged
% % before previous job terminates;
% % or an application can burst its checkpoint to burst buffer
% % with extremely high speed (4.4-17.8 TB/s on Trinity);
% % upon termination, application data is also able to drain off
% % asynchronously to external PFS.
% % When utilizing burst buffer in this primary scenario (\textit{checkpoint \& restart}),
% % bursty application I/O operations can thus be aggregated and absorbed into burst buffers.
% % This makes it possible to shift computations that follows I/O bursts to an earlier moment
% % while burst buffer takes charge of moving potentially TB-level volumes of data.
% % There are more use cases for burst buffer nodes.
% % Among them \textit{data cache} could be equally important to enhance the responsiveness
% % of applications by improving the perceived I/O bandwidth\cite{TrinitySystem}.
% % For example, shared object library or read-only configuration files could be
% % cached on burst buffer nodes;
% % lists of input files specific to a group of compute nodes allocated to
% % a particular application could be loaded to burst buffer prior to execution;
% % Economical solid-state disks as a tier of burst buffer could also be used as
% % out-of-core complement to insufficient main memory\cite{Romanus:CORR:15},
% % working place for data analysis (reductions, feature extraction compression etc.)
% % and visualization\cite{TrinitySystem}.


%2. Use cases of burst buffer
Regardless of the specific deployment, burst buffer nodes essentially augment
the I/O stack with an intermediate productive offloading layer, 
which can benefit jobs in multiple ways.
As discussed in some preliminary surveys\cite{apex-workflow} ,
jobs' latest checkpoints can be pre-staged
before the previous job terminates;
jobs can burst checkpoint to burst buffer
with extremely high speed (4.4-17.8 TB/s on Trinity);
upon termination, jobs are also able to drain off output data
asynchronously to the external parallel file system (PFS).
The bursty I/O operations can thus be aggregated and absorbed into the burst buffer,
which shifts the computation that follows the I/O bursts to an earlier moment,
while burst buffer nodes take charge of moving potentially TB-level volumes of data.
There are more use cases for burst buffer nodes.
Among them, \textit{data cache} is equally important to enhance the responsiveness
of applications by improving the perceived I/O bandwidth\cite{TrinitySystem}.
Shared object library or read-only configuration files are cached into burst buffer;
the list of input files specific to a group of compute nodes allocated to
a particular job are loaded into burst buffer prior to the execution.
Economical solid-state disks as a tier of burst buffer could also be used as
out-of-core complement to insufficient main memory\cite{Romanus:CORR:15},
or working place for data analysis (reductions, feature extraction compression etc.)
and visualization\cite{TrinitySystem}.
