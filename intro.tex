\section{Introduction}

In the field of high performance computing (HPC), 
system performance is no longer throttled only by computation capability,
but also by the ever-increasing I/O gap
between computational resources and disk-based storage technologies.
For example, on the early-state Trinity system~\cite{TrinitySystem}, the I/O nodes between
the compute nodes and the external storage can deliver transfer speed at 810 GB/s,
which is still far from the target of 60 TB/s for the exascale computers\cite{Shalf:HPCCS:2010}.
The gap is critical because scientific applications typically exhibit
``bursty" I/O patterns,
resulting from application's defensive I/O strategy
and the need  for subsequent data processing\cite{Carns:MSST:2011, Kim:PDSW:2010, Latham:CSD:2012, Naik:ICPPW:2009, Dennis:CUG:2009}. 
When multiple applications attempt to access the storage system simultaneously, 
it is easy to saturate
the I/O bandwidth, leading to severe performance degradation.


%3. Very high level intro to burst buffer
Extensive research have been conducted to improve I/O performance on HPC systems from
both hardware and software layers.
We have witnessed a tremendously growing use of flash memory based storage in recent years.
Recently, burst buffer (i.e., a solid state disk based or flash based cache tier)
is proposed to absorb and spread out
the ``bursty" application I/O patterns\cite{Bent:HBP:2011, Grider:EXA:2010}.
% This new cache tier is capable of absorbing and spreading out 
% the ``bursty" application I/O pattern\cite{Bent:HBP:2011, Grider:EXA:2010}.
Studies have demonstrated that the perceived I/O
bandwidth can be significantly improved on burst-buffer-enabled systems\cite{Liu:MSST:2012}.
As such, HPC users are encouraged to explicitly request burst buffer resources at job submission 
on burst-buffer-enabled systems\cite{apex-workflow}.

%4. Our work, a new batch scheduler
In this paper, we propose \textit{Cerberus}\footnote{In Greek mythology,
Cerberus is a monstrous three-headed dog (three-phase scheduler),
who guards the gate of the underworld to the earth (HPC system),
preventing the dead (jobs) from leaving (running).},
a novel batch scheduler for HPC systems equipped with burst buffer. 
We propose a three-phase job model.
%based on the critical use cases of burst buffer, including
%application checkpoint restart and staging in/out data. 
The lifetime of a user job is divided into three logic phases:
\textit{stage in}, \textit{running}, and \textit{stage-out}.
Burst buffer is used for different purposes in these phases.
Unlike existing batch schedulers that make scheduling decision for each job upon its submission,
Cerberus participates in all of the three phases.
By dividing the lifetime of user jobs into three phases,
Cerberus adopts different optimization strategies for 
making scheduling decision based on job requirement at each phase.
We conduct a series of event driven simulations to evaluate our design. 
As shown later on,
the preliminary results demonstrate that the three-phase design of Cerberus 
not only boosts system responsiveness, but also improves job performance. 
Specifically,
Cerberus accelerates job execution, reduces job response time 
and improves the average system throughput.
Put together, we make three key contributions in this work:
\begin{enumerate}
        \item    We propose a three-phase job model (i.e., stage-in, running, and stage-out)
                to facilitate job scheduling in a fine granularity.
        \item    We present Cerberus, a three-phase burst buffer aware scheduler
                for HPC systems equipped with burst buffer.
                Our design consists of several key optimization strategies for
                improving the performance of user jobs throughout different job phases.
        \item    We develop an event driven simulator called BBSim
                for simulating and evaluating Cerberus on a burst buffer enabled Blue Gene system
                using authentic job trace. The simulator is open source and
                freely available to the community.
\end{enumerate}


In the remainder of this paper, Section 2 presents background and 
how existing batch scheduler handle burst buffer on HPC systems.
Section~\ref{Sec:Model} elaborates our three-phase job model.
Section~\ref{Sec:Scheduler} describes the design of Cerberus.
Section~\ref{Sec:Simulation} describes the design of BBsim, the event driven scheduling simulator.
Section~\ref{Sec:Experiments} evaluates Cerberus by comparing it with various alternatives.
Section~\ref{Sec:Conclusion} concludes this paper.

